\documentclass[11pt, letterpaper]{article}
\usepackage[margin=0.6in]{geometry}
%\usepackage{mathpazo}
\usepackage{amsmath}
\usepackage{multicol}

% ------------------------------ CUSTOMIZATION ------------------------- %

\setlength\parindent{0pt}

\title{\textbf{EECS 2210 Notes}}
\author{Richard Robinson}

% -------------------------------- DOCUMENT ---------------------------- %

\begin{document}

\maketitle

\begin{multicols*}{2}

\section{Semiconductors}
The hole density $p$ and electron density $n$ are related by \begin{equation}
    np = n_i^2
\end{equation}
where $n_i$ is the number of electrons per unit volume.

\bigskip
In charge carriers, the velocity of electrons and holes are given by \begin{equation}
    \mathbf v_e = - \mu_n \mathbf E \quad\text{and}\quad \mathbf v_h = \mu_p \mathbf E
\end{equation}
where $\mu$ is the mobility in $\text{cm}^2/(\text{V}\cdot\text{s})$ and $E = V/L$.

\bigskip
Consequently, the current in a carrier is defined as \begin{equation}
    I = -\rho A v_d \quad\text{for}\quad \rho = nq
\end{equation}
such that the current density $J_n = I/A$. The total density is given by \begin{equation}
    J_{tot} = q(\mu_n n + \mu_p p)E
\end{equation}

\end{multicols*}



\end{document}
