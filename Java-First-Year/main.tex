\documentclass[letterpaper, openany, justified]{tufte-book}
\usepackage[utf8]{inputenc}
\usepackage[english]{babel}
\usepackage{amsmath}
\usepackage{amssymb}
\usepackage{listings}
\usepackage{empheq}
\usepackage[dvipsnames]{xcolor}

\setlength{\parindent}{0pt}
\title{\\Java I Exam \\Preparation}
\author{Richard Robinson}

\newcommand{\cd}[1]{\lstinline{#1}}
\definecolor{pblue}{rgb}{0.13,0.13,1}
\definecolor{pgreen}{rgb}{0,0.5,0}
\definecolor{pred}{rgb}{0.9,0,0}
\definecolor{pgrey}{rgb}{0.46,0.45,0.48}
\definecolor{ppurple}{rgb}{0.6,0,0.35}

\lstset{language=java}
\lstset{%
  aboveskip=3mm,
  belowskip=3mm,
  basicstyle={\small\ttfamily},
  captionpos=none,
  keywordstyle=\bfseries\color{magenta},
  commentstyle=\color{pgreen},
  stringstyle=\color{pred},
  breaklines=true,
  breakatwhitespace=true,
  tabsize=4,
  morekeywords={String}
  %moredelim=[il][\textcolor{pgrey}]{$$},
}

\begin{document}
\maketitle
\setlength{\parindent}{0pt}
\begin{fullwidth}

\chapter{GUIs}

\section{Widgets}
There are several classes in the \lstinline{java.swing.*} toolkit, with the syntax \begin{lstlisting}
    JComponent component = new Widget(String); // adds Widget
\end{lstlisting}
where \lstinline{Widget} is the name of the widget. Interacting with such cases an event object to be created, handled by creating a listener for specific events. Such listener must be registered per widget, each causing different events. Event listeners must implement the corresponding interface and methods. The main method for a GUI application is
\begin{lstlisting}
    ClassName frame = new ClassName();
    frame.pack();
    frame.setVisible(true);
\end{lstlisting}
The \lstinline{JButton, JCheckBox, JComboBox, JRadioButton} classes fire \lstinline{ActionEvent} event objects, implement the interface \lstinline{ActionListener} , and registered with the \lstinline{addActionListener} method. For all other classes, the \lstinline{MouseEvent} events are fired, the \lstinline{MouseListener} interface is implemented, and uses the \lstinline{addMouseListener} method. Typically, the \lstinline{app} calls the \cd{frame} which in turn calls the \cd{panel}.

\section{Containers}
The main \lstinline{JFrame} class represents a GUI window with title bar, resizable border, and border buttons. Apps extend this class to customize  it using \begin{lstlisting}
    public class ClassName extends JFrame implements ActionListener
\end{lstlisting}
The \cd{JPanel} class is used to arrange widgets, and can also contain other panels and can be used to draw custom shapes. The \cd{paintComponent} method is called to redraw elements and can be overridden to create custom appearances; it is called by \cd{repaint()}. The listeners are used as follows:
\begin{lstlisting}
    public void actionPerformed(ActionEvent e) {
        Object src = e.getSource();
        switch (source == case); // switch-case conditional
    }
\end{lstlisting}

\section{Layout Managers}
There are several types of layout managers, including:
\begin{itemize}
    \item \cd{FlowLayout} is arranged linearly and flows to next line if needed. It is based on the preferred size.
    \item \cd{BorderLayout} adds components using cardinal directions and ignores preferred size.
    \item \cd{GridLayout} arranged components in a grid and ignores preferred size.
    \item \cd{BoxLayout} is similar to \cd{FlowLayout} with advanced options, and is preferred size.
\end{itemize}
A layout manager is used via a class which appears as follows:
\begin{lstlisting}
    JButton btn;
    public ClassName() { super("LayoutName");
        btn = new JButton("label");
        btn.addActionListener(this);
        JPanel panel = new JPanel();
        panel.setLayout(new LayoutName());
        panel.add(btn, LayoutName.methodName);
    }
\end{lstlisting}

\chapter{Definitions}

\section{Aggregation}
Aggregation represents a \textit{has a} relationship between two classes. A class is an aggregate if it has an attribute of a non-primitive type. This works as given by:
\begin{lstlisting}
    public class ClassName {
        private CustomType var; // attributes
        public ClassName(Type var) { this.var = var; } // constructor
    }

    // using the class in Main
    CustomType var = new CustomType(values);
    ClassName name = new Classname(vars);
\end{lstlisting}
With this code, you can call the attributes of the class via \cd{name.var}.
\section{Classes \& Objects}
Classes are used to define templates, and objects to instantiate classes. Objects are created and methods on objects are called. A template has common attributes (nouns) and behaviors (verbs). Each instance of a class has specific attribute values. An example is
\begin{lstlisting}
    public class Point {
        float x, y; // attributes from another class
        Point(float x, float y) { this.x = x; this.y = y } // constructor
        Type getFunc() { return z } // action returning value
        void actionName(Type var) { this.x = z; } // action modifying values
    }
\end{lstlisting}
Then, in the main class, the object can be called via:
\begin{lstlisting}
    Point p1 = new Point(1, 2); // creating new instance of object
    p1.getFunc(); // calling object without pars
    p1.actionName(var1); // calling object with pars
\end{lstlisting}

\section{References}
The \cd{this} reference replaces the generic variable in a class with a version specific to the variable that is called. It is used when the attributes share names with the constructor parameters so as to disambiguate; \cd{this.par} references the attribute version of \cd{par}, not the parameter version. That is,
\begin{lstlisting}
    float x;
    ClassName(float a) { x = a; }
\end{lstlisting}
is equivalent to
\begin{lstlisting}
    float x;
    ClassName(float x) { this.x = x; }
\end{lstlisting}
Consequently, a mutator method changes values of the attributes to the object it is referencing via the notation \cd{var.changeAtt(par)}. As well, accessor methods use the result of a computation on its attributes, \cd{Type Att = var.getAtt()}.

\chapter{Inheritance}

\section{Subclasses}
A child class may be described as an extension of a parent class. The former inherits all the features of the latter and can implement new features for its particular purpose. The child is a subclass of the parent superclass. When \cd{child} inherits from \cd{parent}, every feature of the latter is in the former and \cd{child extends parent}. Specifically,
\begin{lstlisting}
    public class Child extends Parent { }
\end{lstlisting}
There are numerous types of access:
\begin{itemize}
    \item \cd{public}: all classes can access this feature
    \item \cd{private}: only accessible to the class it is in
    \item \cd{protected}: same as \cd{private} in addition to its derived children.
\end{itemize}

\section{Child Methods}
The child sometimes requires a method to modify or add new feature using \cd{@Override}, in which the child keeps the parent's signature and return type. The child can access the parent's constructor features via
\begin{lstlisting}
    super(pars) // first line in child constructor
\end{lstlisting}
and method features via
\begin{lstlisting}
    super.methodName(pars)
\end{lstlisting}
Specifically, \cd{super()} extends the behavior of the method. An example is given by:
\begin{lstlisting}
    public class ClassA { public void save() { } }

    public class ClassB extends ClassA {
        private Object varB;
        @Override
        public void save() { super.save(); save(varB); }
    }
\end{lstlisting}

\section{Obligatory Methods}
The \cd{Object} class defines methods applicsable to and required by all Java classes. To ensure all classes have these obligatory methods, all classes implicity extend this class. Some obligatory methods include:
\begin{itemize}
    \item \cd{String .toString()}: The Object implementation outputs memory address
    \item \cd{boolean .equals()}: Compares memory addresses.
    \item \cd{int .hashCode()}: Determines location in hash Collections.
\end{itemize}
A program intended to handle \cd{parent} objects will also be able to handle \cd{child} objects with no modification. Both of the following are valid:
\begin{lstlisting}
    Parent varOne = new Parent();
    Parent varTwo = new Child();
    varTwo instanceof Child; // true
\end{lstlisting}
Polymorphism is the ability of a method to take on various forms, such as arguments of different types. It occurs when early binding targets a method in parent and late binding targets the method with the same signature in child.

\chapter{Classes}

\section{Casting \& Objects}
It is neccessary to cast or bind methods to convert them to the proper Object which does have such method so as to not fail; for example,
\begin{lstlisting}
    Parent var = new Child ()
    if (var instanceof ObjectName) {
        name = ((ObjectName) var).methodName
    }
\end{lstlisting}
Early binding occurs at compile time and verifies the method in the class, and late binding occurs at run time only when explict inheretence is used and determines \cd{var} points to the object and calls the method in the class instead.

\section{Abstraction}
Interfaces define only method signatures, of which methods have no implemented body so as to allow the implementer to define class requirements to others. That is, an interface is a group of related methods iwth empty bodies, appearing as:
\begin{lstlisting}
    interface interfaceName {
        void methodOne(Type args);
        void methodTwo(Type args);
    }
\end{lstlisting}
Then, the interface is implemented via
\begin{lstlisting}
    public class SpecificName implements interfaceName {
        Type args;
        void methodOne(Type args) { this.args = args; }
        void methodTwo(Type args) { this.args = args; }
    }
\end{lstlisting}
The implementation must include all the methods of the interface, else the program will fail. Consequently, abstract classes (\cd{public abstract class Name}) implement only some methods and allow implementers to implement some methods and define others. Lastly, in classes, all methods are implemented. The \cd{final} tag prevents its class, method, or variable from being extended, overridden, or changed.


\end{fullwidth}
\end{document}
