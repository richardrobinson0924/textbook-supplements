\documentclass[10pt]{extarticle}
\usepackage[letterpaper, margin=55pt]{geometry}
\usepackage{multicol}
\usepackage{amsmath,amsthm,amssymb}
\usepackage{mathpazo}

\setlength\columnsep{35pt}
\setlength{\parindent}{0pt}

\begin{document}
\begin{multicols*}{2}

%--------------------------------------------------------------------
% Thermodynamics
%--------------------------------------------------------------------

{\large\textit{Thermodynamics}}
\bigskip

When heating a system, $S$ increases where \begin{equation}
  \Delta S_u = \Delta S + \Delta S_s > 0
\end{equation}
and $\Delta \sum v > 0 \iff \Delta S > 0$. As well, $S \propto T \propto r \propto V \propto n \propto P^{-1}$ where $n$ is the moles the side to which equilibrium moves to has.

\bigskip
The Born Haber cycle is used to find $H_f$ of an ionic compound $MX$,
\begin{equation}
  \Delta H_f = \textstyle\sum H \qquad H_B = \textstyle\frac{1}{2} B
\end{equation}
where the sub-reactions are given by
\begin{align}
  \Delta H_s: \qquad & M(s) \to M(g) \nonumber\\
  IE: \qquad & M(g) \to M^+ + e^- \nonumber\\
  \textstyle\frac{1}{2} B: \qquad & \textstyle\frac{1}{2} X_2 \to X \\
  -EA: \qquad & X + e^- \to X^- \nonumber\\
  - \Delta H_l: \qquad & M^+ + X^- \to MX \nonumber
\end{align}
The change in $X : \{ S, G, \Delta H_f \}$ is
\begin{equation}
  \Delta X^\ominus = \Delta \textstyle\sum v X^\ominus
\end{equation}
If $G < 0$ the reaction is spontaneous.

%--------------------------------------------------------------------
% Equilibrium
%--------------------------------------------------------------------

\vspace{7mm}
{\large\textit{Equilibrium}}
\bigskip

The equilibrium constant is given by \begin{equation}
  K_c \sim K_p = K_c RT^{\Delta v_h}
\end{equation}
where $K_p$ is for gaseous reactions. Reactions are homogeneous iff it is the same phase.

\bigskip
To calculate the eq. concentrations, construct an ICE table where \begin{equation}
  [A_i]_{eq} = [A_i] \mp ax
\end{equation}
More reactants are formed in a reaction where
\begin{equation}
  [A] \uparrow \iff [B] \downarrow \iff Q > K
\end{equation}
Exothermic reactions produce heat such that if temperature increases as well, more reactants are formed.

\bigskip
For a reaction $A(s) \leftrightharpoons bB(aq)$, the constant
\begin{equation}
  K_sp = [B]^b \iff x = (K_{sp}/\Pi b^b)^{1/\Sigma b}
\end{equation}

%--------------------------------------------------------------------
% Materials Science
%--------------------------------------------------------------------

\bigskip
{\large\textit{Materials Science}}
\bigskip

The density of a unit cell is given by
\begin{equation}
  \rho = \frac{nM}{a^3 N_a} \qquad a = \left\{ 2r, \; 4r/\sqrt 3, \; 2 \sqrt 2 r  \right\}
\end{equation}
for sc, bcc, and fcc respectively. The packing efficiency is given by $n V_{\text{sph}} / a^3$.

\bigskip
In semiconductors, temperature and impurities are proportional to conductivity, and opposite for conductors.

\bigskip
Thermoplastic polymers melt and deform upon heating. The DP is $\overline M / M_{m}$ and the average molecular weight is
\begin{equation}
  \overline M_n = \frac{\Sigma MN}{\Sigma N} \qquad \overline M_w = \frac{\Sigma M^2 N}{\Sigma MN}
\end{equation}
where $n_\text{chains} = m N_a / \overline M$. Polymers are linear, branched, and crosslinked.

\bigskip
The former two are connected by non-bonded interactions and can be easily recycled, and the latter by covalent bonds.

\bigskip
Linear polymers form crystals more easily and thus become liquid when heated.

%--------------------------------------------------------------------
% Miscellaneous
%--------------------------------------------------------------------

\vspace{7mm}
{\large\textit{Miscellaneous}}
\bigskip

A ketone is --C(=O)--; amides, carboxylics, aldehydes, and esters are KN--, KOH, KH, and KO--.

\bigskip
For stoichiometric problems, use
\begin{equation}
  n_1/v_1 = n_2/v_2
\end{equation}
where $n = m/M = CV$. For an equilibrium reaction, the constant is $K_c$ and only includes gaseous or aqueous compounds.

\bigskip
The partial pressure of a gas is
\begin{equation}
  P_i = X_i P \qquad X_i = n_i / \textstyle\sum n \qquad P = \textstyle\sum P_i
\end{equation}
For a reactant $A$ dissociated $\delta \%$, then \begin{equation}
  P_A = (1-\delta)x \qquad P_B = (v_b \delta / v)a x
\end{equation}
where the mole fraction is
\begin{equation}
  x_i = m_i M / m M_i = n_i / n
\end{equation}

The force in liquids is proportional to BP, viscosity, number of OH$^-$ ions, H, and inversely proportional to $P$ and $T$.

\end{multicols*}{2}
\end{document}
