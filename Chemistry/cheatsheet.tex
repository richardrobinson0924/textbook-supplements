\documentclass{article}
\usepackage[letterpaper, margin=55pt]{geometry}
\usepackage{multicol}
\usepackage{amsmath,amsthm,amssymb}
\usepackage{mathpazo}
\usepackage{fancyhdr}

\pagestyle{fancy}
\fancyhf{}
\lhead{CHEM 1100 Exam Cheatsheet}
\rhead{Richard Robinson}
\setlength\columnsep{25pt}
\setlength{\parindent}{0pt}

\begin{document}
\begin{multicols*}{2}


For stoichiometric problems, use
\begin{equation}
  n_1/v_1 = n_2/v_2
\end{equation}
where $n = m/M = CV$. Lewis dot structures are created by:
\begin{enumerate}
  \item Find the sum of valence electrons
  \item Connect the atoms with single bonds
  \item Create multiple bonds and lone pairs for an octet
\end{enumerate}
For an equilibrium reaction, the constant is
\begin{equation}
  K_c = \textstyle\prod [B]_{eq}^b / \textstyle\prod [A]_{eq}^a = \textstyle\prod K_i
\end{equation}
and only includes gaseous or aqueous compounds. For gaseous reactions, then the constant is
\begin{equation}
  K_p = K_c RT ^{\Delta v_g} \iff P_x \equiv [X]
\end{equation}
Reactions are homogeneous iff all constituents are the same phase.

\bigskip
To calculate the equilibrium concentrations, construct an ICE table such that
\begin{equation}
  [X_i]_{eq} = [A_i] \mp ax
\end{equation}
As well, more reactants are formed if
\begin{equation}
  [A] \uparrow \iff [B] \downarrow \iff Q > K
\end{equation}
and $P^{-1} \propto n$ where $n$ is the moles the side to which equilibrium moves to has.

\bigskip
Exothermic reactions produce heat such that if temperature increases as well, more reactants are formed.

\bigskip
For a reaction $A(s) \leftrightharpoons bB(aq)$, the constant
\begin{equation}
  K_sp = [B]^b \iff x = (K_{sp}/\Pi b^b)^{1/\Sigma b}
\end{equation}
Gibbs free energy is defined as
\begin{equation}
  \Delta G^\ominus = -RT \ln K = H - TS = -T \Delta S_u
\end{equation}
When heating a system $S$ increases such that \begin{equation}
  \Delta S_u = \Delta S + \Delta S_s >0
\end{equation}
where $\Delta S_s = - \Delta H/T$ for a spontaneous reaction. If $\Delta \sum v > 0 \iff \Delta S > 0$.

\bigskip
The change in $\{S,G,\Delta H_f\}$ is
\begin{equation}
  \Delta X^\ominus = \Delta \sum v X^\ominus
\end{equation}
If $G < 0$ the reaction is spontaneous. Entropy is proportional to $T,r,V,n,P^{-1}$.

\bigskip
The Born Haber cycle is used to find $H_f$ of an ionic compound $MX$,
\begin{equation}
  \Delta H_f = \sum H \qquad H_B = \textstyle\frac{1}{2} B
\end{equation}
where
\begin{align}
  \Delta H_s: \qquad & M(s) \to M(g) \\
  IE: \qquad & M(g) \to M^+ + e^- \\
  \textstyle\frac{1}{2} B: \qquad & \textstyle\frac{1}{2} X_2 \to X \\
  -EA: \qquad & X + e^- \to X^- \\
  - \Delta H_l: \qquad & M^+ + X^- \to MX
\end{align}
The density of a unit cell is given by
\begin{equation}
  \rho = \frac{nM}{a^3 N_a} \qquad a = \left\{ 2r, \; 4r/\sqrt 3, \; 2 \sqrt 2 r  \right\}
\end{equation}
for sc, bcc, and fcc respectively. The packing efficiency is given by $n V_{\text{sph}} / a^3$.

\bigskip
In semiconductors, temperature is proportional to conductivity, and oppoiste for conductors.

\bigskip
The force in liquids is proportional to BP, viscosity, number of OH$^-$ ions, H, and inversely proportional to $P$ and $T$.

\bigskip
Thermoplastic polymers melt and deform upon heating. The DP is $\overline M / M_{m}$ and the average molecular weight is
\begin{equation}
  \overline M_n = \frac{\Sigma MN}{\Sigma N} \qquad \overline M_w = \frac{\Sigma M^2 N}{\Sigma MN}
\end{equation}
where $n_\text{chains} = m N_a / \overline M$. Polymers are linear, branched, and crosslinked.

\bigskip
The former two are connected by non-bonded interactions and can be easily recycled, and the latter by covalent bonds.

\bigskip
Linear polymers form crystal more easily and thus become liquid when heated.

\bigskip
The partial pressure of a gas is
\begin{equation}
  P_i = X_i P \qquad X_i = n_i / \Sigma n \qquad P = \sum P_i
\end{equation}
For a reactant $A$ dissociated $\delta \%$, then \begin{equation}
  P_A = (1-\delta)x \qquad P_B = (v_b \delta / v)a x
\end{equation}
where the mole fraction is
\begin{equation}
  x_i = m_i M / m M_i = n_i / n
\end{equation}

\end{multicols*}{2}
\end{document}
