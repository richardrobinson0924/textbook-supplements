\documentclass[answers]{exam}
%\documentclass{exam}
\usepackage[utf8]{inputenc}
\usepackage{amsmath,amsthm,amssymb}
\usepackage{multicol}

\title{CHEM 1100 Practice Exam}
\author{Richard Robinson}

\begin{document}
\maketitle

\section{Multiple Choice}

\begin{questions}

\question Which of the following is an example of a homogenous equilibrium?

\begin{choices}
 \choice $\mathrm{MgCO_3(s) \leftrightharpoons MgO(s) + CO_2(g)}$
 \choice $\mathrm{NaCl(s) \leftrightharpoons Na^+(aq) + Cl^-(aq)}$
 \CorrectChoice $\mathrm{3H_2(g) + N_2(g) \leftrightharpoons 2NH_3(g)}$
 \choice $\mathrm{C(s) + CO_2(g) \leftrightharpoons 2CO(G)}$
 \choice None of the above.
\end{choices}

\question A system in chemical equilibrium is \emph{not} characterized by one of the following:

\begin{choices}
  \choice Dynamic interconversion between reactants and products
  \choice No macroscopic changes
  \CorrectChoice Unaffected by changes in temperature
  \choice Unaffected by addition of catalyst
  \choice None of the above
\end{choices}

\question Which is \emph{false} about the first law of thermodynamics?

\begin{choices}
  \choice All energy change in a chemical reaction is in the form of heat
  \choice The enthalpy of the universe is zero
  \choice $\Delta E = \Delta E_\text{sys} + \Delta E_{\text{sur}}$
  \choice Energy cannot be created nor destroyed
  \choice The combined amount of matter in the universe is constant
\end{choices}

\question Which is \emph{false} about the second law of thermodynamics?

\begin{choices}
  \choice In any spontaneous process, entropy of the universe increases
  \choice In any spontaneous process, entropy of the system increases
  \choice The entropy of the surroundings can increase or decrease
  \choice $\Delta S_u = \Delta S + \Delta S_s$
  \choice The entropy of the universe is positive for a spontaneous process
\end{choices}

\question If $\mathrm{Ba(NO_3)_2}$ is added to $\mathrm{BaSO_4}$, the solubility of the latter:

\begin{choices}
  \choice is unaffected
  \choice is unpredictable
  \choice decreases
  \choice increases
\end{choices}

\end{questions}

\section{Short Answer}

\begin{questions}

\question Calculate $\Delta S$ for the reaction $\mathrm{2NO_2 \to 2N_2 + O_2}$. Note $\Delta S = \{ 240, 191.5, 205 \}$, respectively.
\answerline[108]

\question What is the total number of lone pairs in $\mathrm{NCl_3}$?
\answerline[10]

\question In manufacturing steel, carbon is likely to be a (?) impurity because it is (?) than iron.
\answerline[int., smaller]

\question In the reaction $\mathrm{A(g) + 3 B(\ell) \to 3 C(g) + 7 D(g)}$, what are the exponents in the denominator of the equilibrium expression?
\answerline[1; 0]

\question Given the heat of formation values $\{-103.8, 0, -393.5, -285.8 \}$, calculate the heat of reaction for $\mathrm{A(g)}$ $\mathrm{ + B(g) \to C(g) + D(\ell)}$.
\answerline[$\mathbf{-2.22 \cdot 10^3}$]

\question What is the molar solubility of $\mathrm{CaF_2}$ if $K_{sp} = 3.9 \cdot 10^{-11}$?
\answerline[$\mathbf{2.14 \cdot 10^{-4}}$]

\question What is the pH of a 0.15 M NaOH solution?
\answerline[13.18]

\question Which type of solid is most densely packed?
\answerline[fcc]

\question What element (Ga, Si, Al, Ar) would be added to Ge to produce an \textit{n}-type conductor?
\answerline[Ar]

\question The volume of a gas is 650 mL at STP. What volume will it occupy at freezing point and 950 torr?
\answerline[520]

\end{questions}

\newpage
\section{Long Answer}

\begin{questions}

\question Use the Born Haber cycle to determine the lattice energy of KF (s) from the following data:
\begin{align*}
  \Delta H_f^\ominus &= - 567.3 & \Delta H_{sub} [K(s)] &= 89.24 & \Delta H_{dis} [F_2(g)] &= 159 \\
  IE[K(g)] &= 418.9 & EA[F(g)] &= -328
\end{align*}
\begin{solutionbox}{3.2in}
  The Born Haber Cycle is given by \begin{equation*}
    \Delta H_f^\ominus = \sum \Delta H^\ominus
  \end{equation*}
  in which the enthalpies are given by
  \begin{align*}
    & \text{Formation} & K(s) + \textstyle\frac{1}{2} F_2(g) &\to KF(s) & \Delta H_f^\ominus &= -567.3 \\
    & \text{Sublimation} & K(s) &\to K(g) & \Delta H_s &= 89.24 \\
    & \text{Ionization} & K(g) &\to K^+(g) + e^- & \Delta H_i &= 418.9 \\
    & \text{Dissociation} & \textstyle\frac{1}{2} F_2 (g) &\to F(g) & \Delta H_d &= 0.5 \cdot 159 \\
    & \text{Affinity} & F(g) + e^- &\to F^- (g) & \Delta H_e &= -328
  \end{align*}
  Therefore, the cycle is \begin{equation*}
    -567.3 = 89.24 + 418.9 + 0.5 \cdot 159 - 328 - \Delta H_l \implies \Delta H_l = 827
  \end{equation*}
  The lattice energy is thus 827 kJ/mol.
\end{solutionbox}

\bigskip
\question For the reaction $H_2(g) + I_2(g) \leftrightharpoons 2HI(g)$, the constant $K = 57$ at 700K. If 1 mol $H_2$ reacts with 1 mol $I_2$ in a 10L vessel at 700K, what is the molar composition at equilibrium?
\begin{solutionbox}{3.05in}
  The initial concentrations are given by $[H_2] = [I_2] = n/V = 0.1 \text{ M}$. An ICE table is next constructed to determine the concentrations at equilibrium:
  \begin{center}
  \begin{tabular}{llll}
    R & $H_2$ & $I_2$ & $2HI$ \\[1mm]
    I & 0.100 & 0.100 & 0 \\[1mm]
    C & $-x$ & $-x$ & $+2x$ \\[1mm]
    E & $0.100 - x$ & $0.100 - x$ & $2x$
  \end{tabular}
  \end{center}
  The change in concentration is therefore given by \begin{equation*}
    K_c = 57 = \frac{(2x)^2}{(0.1-x)(0.1-x)} \implies x = \begin{cases}
      0.0791 \text{ M} \\ 0.136 \text{ M}
  \end{cases}
\end{equation*}
  Thus, the valid solution is $x = 0.0791$ so the concentrations at equilibrium are \begin{equation*}
    [H_2]_{eq} = [I_2]_{eq} = 0.1 - x = 0.0209 \text{ M} \quad\text{and}\quad [HI]_{eq} = 2x = 0.1582 \text{ M}
  \end{equation*}
\end{solutionbox}

\newpage

\question A mixture of 1.57 mol $N_2$, 1.92 mol $H_2$, and 8.13 mol $NH_3$ is mixed in a 20L vessel at 500K. At this temperature, $K_c = 1.7 \cdot 10^2$ for $N_2 + 3H_2 \leftrightharpoons 2NH_3$. Is such mixture at equilibrium? If not, what is the direction of the net reaction?

\begin{solutionbox}{1.95in}
  The initial concentrations are given by \begin{equation*}
    [N_2] = 0.0785 \qquad [H_2] = 0.0960 \qquad [NH_3] = 0.406
  \end{equation*}
  The reaction quotient is thus \begin{equation*}
    Q = \frac{[NH_3]^2}{[N_2][H_2]^3} = \frac{(0.406)^3}{(0.0785)(0.0960)^3} = 2.37 \cdot 10^3
  \end{equation*}
  Therefore, $Q > K$ so the mixture is not at equilibrium and the net reaction will proceed leftwards, decreasing the $NH_3$ concentration.
\end{solutionbox}

\end{questions}

\end{document}
