\documentclass{tufte-book}
\usepackage[utf8]{inputenc}
\usepackage[english]{babel}
\usepackage{amsmath,amsthm,amssymb}
\usepackage{tikz}
\usepackage{graphicx}
\graphicspath{{../images/}}

\setlength{\parindent}{0pt}

\title{\\ Chemistry \& \\ Materials \\ Science}
\author{Richard Robinson}

\begin{document}
\frontmatter
\maketitle
%\tableofcontents

\setlength{\parindent}{0pt}

\mainmatter


%%%%%%%%%%%%%%%%%%%%%%%%%%%%%%%%%%%%%%%%%%%%%%%%%%%%%%%%%%%%%%%%%%%%%%
% MAIN DOCUMENT
%%%%%%%%%%%%%%%%%%%%%%%%%%%%%%%%%%%%%%%%%%%%%%%%%%%%%%%%%%%%%%%%%%%%%%

\chapter{Introduction}

\section{Stoichiometry}

All stoichiometric equations for quantities may be derived from \begin{equation}
  n_1/v_1 = n_2/v_2
\end{equation}
The percentage yield is defined as \begin{equation}
  \text{\% yield} = \frac{\text{actual yield}}{\text{theo. yield}} \times 100 \%
\end{equation}

A crucial aspect in confirming stoichiometric results is using dimensional analysis; that is, using the dimensions of each unit in the equation(s) and confirming the final unit has the proper dimensions.
\begin{center}
  \begin{tabular}{ll}
    moles & $n = m/M = CV$ \\
    atoms & $n_{\mathrm{atoms}} = \rho V N_a / M$ \\
    molarity & $C = mn/MV$ \\
    dilution & $n_1 = n_2$
  \end{tabular}
\end{center}

\section{Bonding}

The total number of orbitals is equal to $n^2$, where $n$ is the principle quantum number, wherein each orbital has a maximum of two electrons. The oribitals are filled in the order of \begin{equation}
  1s^2 \, 2s^2 \, 2p^6 \, 3s^2 \, 3p^6 \, 4s^2 \, 3d^{10} \, 4p^6 \, 5s^2 \dots
\end{equation}
This maximum occurs only if all subshells contain one electron originally, known as Hund's rule.

\begin{center}
  \includegraphics[width=1\textwidth]{table}
\end{center}

The formal charge is given by $q_f = n_v - n_l - \frac{1}{2} n_s$ in which $\sum q_f = 0$, where $n_v$ is the total number of valence electrons; $n_l$ is the number of lone pairs; and $n_s$ is the number of electrons shared in bonds.

\bigskip
Hybrid orbitals are dependent on molecular geometry, and filled by $sp^3d^2$. The number of orbitals is equal to the number of electron pairs. The number of sigma and pi bonds is equal to \begin{equation}
  n_\sigma = \sum n_\text{all} \quad\text{and}\quad n_\pi = \sum n_\text{dbl} + 2 \sum n_\text{tri}
\end{equation}

\section{Atoms \& Molecules}
Other properties of the periodic table include \begin{itemize}
  \item Atomic size increases toward the bottom left;
  \item Ionization energy and electronegativity increase towards the top right;
\end{itemize}
Lewis dot structures are created via the following algorithm: \begin{enumerate}
  \item Count the total number of valence electrons in the molecule
  \item Place single bonds between all connected atoms;
  \item Place the remaining valence electrons not accounted for in (2) on individual atoms, specifically as lone pairs whenever possible;
  \item Create multiple bonds as needed for any atoms that do not have a full octet
\end{enumerate}

\chapter{Chemical Equilibrium}

\section{Equilibrium Constants}

A system is said to be at dynamic equilibrium if the rates of both reactions are equal but do not approach zero. For a general chemical reaction, the reaction quotient and equilibrium constant are
\begin{equation}
  Q = \frac{[C]^c [D]^d}{[A]^a [B]^b} \quad\text{and}\quad K_c = \frac{[C]^c_{eq} [D]^d_{eq}}{[A]^a_{eq} [B]^b_{eq}}
\end{equation}
respectively. For reactions which take place in the gas phase, \begin{equation}
  K_p = K_c RT^{\Delta n_g} \iff [X] \equiv P_X = [X]RT
\end{equation}
where $\Delta n_g = c+d-(a+b)$. Reactions are homogeneous iff all constituents are either exclusively gaseous or aqueous. Incidentally, in a heterogeneous reaction, $K$ only includes the compounds in the reaction which are not solid nor liquid. For a series of reactions, \begin{equation}
  K_n = \prod K_i
\end{equation}
The procedure to calculate final concentrations of specific compounds in a reaction $A \leftrightharpoons B$ is as follows:
\begin{enumerate}
  \item For reactants and products, $[A_i]_{eq} = [A_i] \mp ax$, respectively.
  \item Using these concentrations in $K$, solve for $x$.
  \item Substitute $x$ into the original equilibrium concentrations.
\end{enumerate}
\begin{center}
  \begin{tabular}{clll}
    R & $A$ & $A'$ & $B$ \\
    \hline
    I & $[A_i]$ & $[A'_i]$ & $[B_i]$ \\
    C & $-ax$ & $-a'x$ &  $+bx$ \\
    E & $[A]_{eq}$ & $[A']_{eq}$ & $[B]_{eq}$
  \end{tabular}
\end{center}

\section{LeChatelier's Principle}

LeChatelier's principle states that when a system at equilibrium is stressed, it reestablishes itself to avoid such stress; that is,
\begin{itemize}
  \item If the concentration of products is increased or reactants decreased, $Q>K$ and more reactants are formed;
  \item If vice versa, $Q<K$ and more products are formed.
\end{itemize}
Pressure is inversely proportional to the amount of moles the side to which equilibrium moves to has. Note that catalysts have no effect upon the properties of --thermic reactions.

\bigskip
Exothermic reactions produce heat and endothermic reactions absorb heat. If a reaction is exothermic and temperature increases or vice versa, more reactants are formed. Otherwise, more products are formed.

\section{Solubility Equilibria}
For a reaction $A(s) \leftrightharpoons cC(aq) + dD(aq)$, the solubility constant is defined as \begin{equation}
  K_{sp} = [C]^c [D]^d
\end{equation}
and the molar solubility is \begin{equation}
  x = (K_{sp}/c^cd^d)^{1/(c+d)}
\end{equation}
A Brønsted-Lowry acid is a proton $H^+$ donor, and a base a proton acceptor. The equation for the dissociation of a weak acid $HA$ is \begin{equation}
  HA(aq) + H_2O (\ell) \leftrightharpoons H_3O^+ (aq) + A^- (aq)
\end{equation}
snd for a weak base $B$,
\begin{equation}
  B(aq) + H_2O (\ell) \leftrightharpoons BH^+ (aq) + OH^-(aq)
\end{equation}
The acid and base ionization constants are $K_a,K_b = K_s$ respectively such that $[X]_{eq} = [X]_i$ for $X = HA, B$. The Gibbs free energy is related to $K$ by \begin{equation}
\Delta G^\ominus = -RT \ln K = H-TS
\end{equation}

\chapter{Thermodynamics}

\section{The Second Law}
\textsc{A spontaneous reaction} occurs without the need for continuous intervention. When heating a system, its entropy $S$ increases. The second law states that for a spontaneous reaction, \begin{equation}
  \Delta S_u = \Delta S + \Delta S_{sur} > 0
\end{equation}
where $\Delta S_{sur} = - \Delta H/T$. If $\Delta \sum v > 0$, then $\Delta S > 0$. The change in standard entropy is \begin{equation}
  \Delta S^\ominus = \Delta \sum v_i S^\ominus_i
\end{equation}
The change in Gibbs free energy is thus also equal to \begin{equation}
  \Delta G = \Delta H - T \Delta S = -T \Delta S_u
\end{equation}
If $\Delta G > 0$, a reaction is spontaneous and vice versa. The change in standard free energy is \begin{equation}
  \Delta G^\ominus = \Delta \sum v_i G^\ominus_i
\end{equation}
Entropy is proportional to $T, r, V, n$ and inversely proportional to $P$.

\section{Enthalpy}
Heat is expressed as $\Delta E - w$. The specific \& molar heat capacities are defined as \begin{equation}
  q_s = mc \Delta T \quad\text{and}\quad q_m = nC_p \Delta T
\end{equation}
For calorimetry, heat is given by \begin{equation}
  q_c = -(q / \Delta T_i) \Delta T_f
\end{equation}
Enthalpy is defined as \begin{equation}
  \Delta H = \Delta E + \Delta (PV) = q_p - n \Delta H_{\Delta \text{phase}}
\end{equation}
A reaction is exothermic for $\Delta H<0$ and vice versa. A formation reaction is one in which one mole of a compound is formed from its elements such that \begin{equation}
  aA + bB = 1AB \qquad \Delta H^\ominus = \Delta H_f^\ominus [AB]
\end{equation}
Hess's Law states the enthalpy change of a process is independent of path; that is, $\Delta H = \sum \Delta H_i$ meaning that sub-reactions and their enthalpies may be summed together to yield the desired reaction and its enthalpy. The change in standard enthalpy is \begin{equation}
  \Delta H^\ominus = \Delta \sum v \Delta H_f^\ominus
\end{equation}
In thermodynamical equations, always convert to moles from mass or molarity using \begin{equation}
  q = \pm n \Delta H / N_a
\end{equation}
The Born-Haber cycle is used to find the $H_f$ of an ionic compound $MX$, given by \begin{equation}
  \Delta H_f = H_{\text{sub}} + \textstyle{\frac{1}{2}} B_X + IE_M - EA_X + LE
\end{equation}
where $B$ is the bond energy of $X_2$ from $M + \frac{1}{2} X_2 \to MX$, and LE is defined as exothermic.

\chapter{Materials Science}

\section{Crystal Structures}
\textsc{Maximum packing} density is achieved with ccp and hcp configurations. The cubic unit cell may be one of sc, bcc, or fcc. The number of atoms per unit cell is equal to $1 \over 2$ number of face--centered atoms + $1 \over 8$ number of atoms.
\begin{center}
  \begin{tabular}{llll}
    & $n$ & cn & a$_0$ \\
    \hline
    sc & 1 & 6 & $2r$ \\
    bcc & 2 & 8 & $4r / \sqrt 3$ \\
    fcc & 4 & 12 & $2 \sqrt 2r$
  \end{tabular} \phantom{mm}
\end{center}
The density of a cell is given by \begin{equation}
  \rho = \frac{m}{a^3} \quad\text{for}\quad m = \frac{nM}{N_a}
\end{equation}
The packing efficiency of a lattice is defined as $n V_{\text{sph}} / a^3$.

\section{Types of Bonding}
The concept of band theory describes metallic bonding in semiconductors. In metals / conductors, semiconductors, and insulators, the gap between the lower valence and upper conduction bands is negligible, moderate, and large, respectively.

\begin{marginfigure}
\begin{center}
  \includegraphics[width=\textwidth]{ntype} \phantom{mmm}
\end{center}
\end{marginfigure}
\marginnote[-3mm]{The band theory for semiconductors.}

\bigskip
In an n-type semiconductor, donor electrons are promoted easily into the conduction band, and the dopant has more valence electrons than the semiconductor. Likewise, in a p-type, vice versa.

\bigskip
Hydrogen bonding is the strongest intermolecular force, and occurs between a hydrogen atom and a highly electronegative atom. Dipole forces occur for non-symmetrical compounds, and dispersion forces occur in the absence of the former three.

\section{Condensed Phases}
In the liequid state, the force is proportional to boiling point, viscosity, number of OH$^-$ ions, $H$, and inversely proportional to vapor pressure and $T$.

\bigskip
An addition polymerization begins with a molecule breaking down into two free radicals. A radical then attaches to a monomer molecule breaking the double bond, forming a new radical. This recursive procedure when the remaining original radical combines with the new polymer. A condensation reaction is one in which two monomers with functional groups combine to former a monomer and $H_2O$.

\bigskip
A block copolymer has repeated segments of each monomer, and a graft copolymer has segments of one monomer branching off the main chain. Additionally, thermoplastic polymers melt and deform upon heating, conversely of thermosetting ones.

\bigskip
The degree of polymerization is $\overline M / M_{mon}$ and the average molecular weight of such chains is \begin{equation}
  \overline M_n = \frac{\sum MN}{\sum N} \quad\text{and}\quad \frac{\sum M^2 N}{\sum MN} \quad\text{for}\quad n_{\mathrm{chains}} = \frac{m N_a}{\overline M}
\end{equation}

\chapter{Miscellaneous}

\section{Organic Chemistry}

\textsc{Some common} functional groups include
%
\begin{center}
  \begin{tabular}{ll}
    Alcohol & R--OH \\
    Ethers & R--O--R \\
    Amines & N \\
    Carboxylic & R--[C=O]--OH
  \end{tabular} \qquad\qquad
  \begin{tabular}{ll}
    Amides & R--[C=O]--N \\
    Aldehydes & R--[C=O]--H \\
    Ketones & R--[C=O]--R \\
    \phantom{K}
  \end{tabular}
\end{center}
%
LDPE and HDPE are linear and branched versions of polyethylene.

\section{Energetics}
The ionization energy is the energy required for the reaction $A \to A^+ + e^-$. Conversely, electron affinity is the energy required for $B + e^- \to B^-$.

\bigskip
The lattice energy is defined as $U_{lat} = |\Delta H^\ominus_{lat}|$, and is proportional to $q$ and inversely proportional to $r$. This energy occurs in formation reactions $A^+ + B^- \to AB$ when $\Delta H_\ell < 0$.

\section{Gases}

The partial pressure of one gas is \begin{equation}
  P_i = X_i P \quad\text{for}\quad X_i = n_i/\sum n \quad\text{and}\quad P = \sum P_i
\end{equation}
Consequently, or a reactant $A$ dissociated $\delta \%$, then \begin{equation}
  P_A = (1- \delta)x \quad\text{and}\quad P_B = (v_B \delta / v_A)x
\end{equation}
where $x_i$ is the mole fraction, \begin{equation}
  x_i = \frac{m_i}{m} \frac{M}{M_i} = \frac{n_i}{n}
\end{equation}










\end{document}
