\documentclass[oneside, 11pt]{book}
\usepackage[letterpaper, margin=1in]{geometry}
\usepackage[T1]{fontenc}
\usepackage{mathpazo}
\usepackage{amsmath, amssymb}
\usepackage{minted}
\usepackage{beramono}

\usemintedstyle{xcode}
%\setminted[matlab]{fontfamily=fdm, fontsize=\footnotesize}
\setminted{fontsize=\footnotesize}

\setlength{\parindent}{0pt}
\title{\Huge{\textbf{Project Management \& Economics}}}
\author{Richard Robinson 215353287}

\begin{document}
\maketitle
\tableofcontents
\setlength{\parindent}{0pt}

\chapter{Time Value of Money}

\section{Types of Interest}

\emph{Nominal} or simple interest means the interest earned per interest period is exclusively a linear function of the principle amount of currency. This is opposite \emph{compound} interest, which means the interest is added to the principal amount at the start of every new interest period and is exponential.

\bigskip
The compounded interest rate per a compound period is given by \begin{equation}
    i = \left(1 + \frac{r}{n} \right)^{nt} - 1 \iff A = P \left(1 + \frac{r}{n} \right)^{nt}
\end{equation}
where $r$ is the nominal interest rate per year, and $n$ is the number of compounding periods per year of $t$ years.

\section{Interest Rates}
The \emph{nominal rate} is the "actual" interest excluding the concept of compounding. This contrasts with the \emph{effective rate} which includes compounding. Depreciation is the reduction of value of an object. For a product costing \$$P$ which in $n$ years is worth \$$S$, then the depreciation rate per year is $1/n$\%.

\bigskip
As well, assuming a linear depreciation then the depreciable asset cost is \$$(P-S)$ such that the annual depreciation is \begin{equation}
    (P-S)/n
\end{equation}

\section{Data Analysis}
Using MATLAB, an income stream array for an investment(s) with a series of returns per period may be created as follows:
\begin{minted}{matlab}
    Stream = [-investment1, return1, etc]; % new stream array
    ROR = irr(Stream); % rate
\end{minted}
where \mintinline{matlab}{ROR} is the value(s) of the roots of the rate of return equation; the IRR is best if returns are in the early months rather than the later. For \emph{fixed} regular payments, the periodic interest rate, number of periods, and size of payment must be known. However for \emph{variable} regular payments, the cash flow and interest rate must be known. In MATLAB, these is given by
\begin{minted}{matlab}
    present_value = pvfix(r/m, mt, p/m); % fixed payments
    present_value = pvvar(Stream, i, Dates); % var payments
\end{minted}
Additionally, the declining balance method calculates the annual depreciation via
\begin{minted}{matlab}
    depr = depstln(P, S, n); % year 1 depreciation
    dbm = depgendb(P, S, n, 2); % depreciation for each year
\end{minted}

\section{Interest Calculations}
MATLAB is also able to calculate different interest rates, as given by the following:
\begin{minted}{matlab}
    rate = annurate(mt, A, P, 0, 0) % find annural rate / 12
    [P, I, bal, A] = amortize(r/m, mt, P) % find monthly payment
\end{minted}
where $P$ is the initial amount, $A$ is the payment per period, and $r$ is the annual rate. Given $P, A, I$ where $I$ is the monthly interest rate, the number of periods $n$ can be found via
\begin{minted}{matlab}
    periods = annuterm(I, A, -P) % periods to pay off loan
    periods = annuterm(I, A, P, F) % periods until FV is reached
\end{minted}

\chapter{Project Management}

\section{Roles of Members}
There are several critical types of members involved in a project, whom's roles are described below:
\begin{itemize}
    \item \emph{Project Manager}: Makes decisions, and monitors and controls the team. They are also in charge of morale via conflict management and reward adjustment.

    \item \emph{General Members}: Do their jobs on schedule and in accordance with the budget. They must also communicate and help resolve problems.

    \item \emph{Functional Managers}: Deliver on resources promised, stay informed on the goings on, and take responsibility for the quality of work.

    \item \emph{Sponsor}: An entity that has the resources to complete the project and leads the development of the project charter.

    \item \emph{Senior Management}: Do not obstruct with the project, but rather reward success and give support when needed. As well, they should respect the project management process and take some responsibility.
\end{itemize}

\section{Definitions}
The typical definition of a \emph{project} is something which delivers a result, often a product or service, in which there is a limited amount of time and money. Additionally, a project usually involves a number of people, and is a unique endeavor.

\bigskip
\emph{Project Management} (PM) itself can be defined as encompassing several steps, including initiating, planning, executing, monitoring, and closing. Specifically,
\begin{itemize}
    \item In the \emph{initiation stage}, the nature and scope of a project are defined in the project charter, which also describes what the project will accomplish.

    \item In the \emph{planning stages}, the project is broken down into smaller modules, the sequence of activities is defined, and the resources are identified.

    \item In the \emph{execution stage}, work is carried out to deliver the product, service, or desired outcome, at which point the project manager assumes full responsibility and control.

    \item In the \emph{monitoring stage}, progress reports and consistent assessments are carried out.

    \item The \emph{closing stage} occurs at the end of the project, in which the manager ensures the client is satisfied with the outcome.
\end{itemize}
Of course, PM should only be implemented upon a project with specific deliverables.

% Establish team goals, roles, and responsibilities
% ↳ Team goals are well articulated
% ↳ Roles & Responsibilities are clear and assigned according to need and capability
%
% Resolve team conflict and negotiate resolution to ensure project completion
% ↳ Astute analysis of reasons for conflict
% ↳ Remains nonjudgemental
% ↳ Negotiates a resolution that is satisfactory to all parties
%
% Identify all limitations of economics and business practices in engineering
% ↳ Takes context into account
%
% Monitor all risks during the life cycle of the project
% ↳ Takes resources into account
%
% Apply economic principles to support decision making
% ↳ Based on economic analysis
% ↳ Use TVM, interest, and comparison methods







\end{document}
