\documentclass[oneside,11pt]{book}
\usepackage[T1]{fontenc}
\usepackage[letterpaper, margin=1in]{geometry}
\usepackage{amsmath,amsthm,amssymb}
\usepackage{mathpazo}
\usepackage{listings}
\usepackage{minted}


\setlength{\parindent}{0pt}
\title{\Huge{\textbf{Assembly \& Verilog}}}
\author{Richard Robinson}

\begin{document}
\maketitle
\setlength{\parindent}{0pt}

\chapter{Using RISC V}

\section{Start of Program}

A typical RISC V or Assembly program starts with the following lines:
\begin{minted}{gas}
	ORG     96		  # declares init address
	DD      42, 100, 19, 2000   # stores values in mem
	addi    x1, x0, 96	  # initializes init address
\end{minted}
The first line contains the \verb$ORG$ command, which declares the initial address. The \verb$DD$ command stores its arguments in memory to be later used by the program. Lastly, the \verb$addi$ command with the \verb$x1, x0$ arguments initializes the registers to the initial address in \verb$ORG$.

\section{Using Registers}
There are two primary methods of loading values into registers;
\begin{minted}{gas}
	addi    x4, x0, 42	  # method 1
	ld      x4, 8(x1)	   # method 2
\end{minted}
In the first method, the value 42 is directly loaded into \verb|x4|. In the second method, the value must first be stored in memory via the \verb$DD$ command. To load the $n$th number of \verb$DD$, \verb$8n(x1)$ is used as $8n + 96$ is the address of the value.

\bigskip
Consequently, values are loaded from registers into memory via
\begin{minted}{gas}
	sd      x19, 88(x1)         # stores reg in mem
\end{minted}
which stores the value contained in \verb$x19$ into memory address \verb|88 + x1|, which in this example would be $88 + 96$.


\end{document}
