\documentclass[oneside, 11pt]{book}
\usepackage[letterpaper, margin=1.3in]{geometry}
\usepackage{amsmath,amsthm,amssymb}

% circuits
\usepackage[format=plain, font=it]{caption}
\usepackage{tikz}
\usepackage[siunitx, american]{circuitikz}
\ctikzset{bipoles/resistor/height=0.15}
\ctikzset{bipoles/resistor/width=0.4}
\ctikzset{bipoles/americaninductor/height=0.15}
\ctikzset{bipoles/americaninductor/width=0.4}
\ctikzset{bipoles/twoport/height=0.5}
\ctikzset{bipoles/twoport/width=0.5}

\title{\Huge{\textbf{Electric Circuits}}}
\author{Richard Robinson, B.Eng. Cand.}
\begin{document}
\maketitle
\tableofcontents
\setlength{\parindent}{0pt}

\chapter{Introduction}

\section{Definitions}
The basic definitions of circuit analysis must first be thoroughly understood. The main units of electrodynamics are voltage and current, defined as \begin{equation}
    v = \frac{dw}{dq} \quad\text{and}\quad i = \frac{dq}{dt}
\end{equation}
respectively. The voltage is precisely the potential difference across an element, and the current is the flow of charge across a branch. Additionally, the power of an element is \begin{equation}
    p = \frac{dw}{dt} = vi \;\;\text{W}
\end{equation}
The passive sign convention (PSC) states an element's voltage is positive if its reference current is in the direction of the voltage drop; that is, from positive to negative. By convention, currents entering a node are negative, and positive if leaving.

\section{Kirchoff's Laws}
The current $i$ in any branch of a circuit can be arbitrarily chosen unless there is a current source $i_s$. In this case, the branch current is in the direction of $i_s$. Kirchoff's current and voltage laws state for any planar circuit, then
\begin{equation}
    \sum_{\text{junc.}} i = 0 \quad\text{and}\quad \sum_{\text{loop}} v = 0
\end{equation}
Conventionally, the direction of each voltage loop is clockwise. Typically, Kirchoff's law equations have the form \begin{equation}
    \sum_{\text{loop}} v_s = \sum_{\text{loop}} i_iR_i \quad\text{and}\quad \sum_{\text{to}} i = \sum_{\text{from}} i
\end{equation}

\section{Simple Analysis Methods}

The voltage across branches in parallel are equal, as is the current across a branch in series. Otherwise, the voltage-series and current-parallel division equations may be used,
\begin{equation}
    v_j = iR_j = \frac{R_j}{R_{eq}} v_s \quad\text{and}\quad i_j = v/R_j = \frac{R_{eq}}{R_j} i_s
\end{equation}
where $v_s$ is the total voltage drop of the series connection and $i_s$ is the current entering the parallel connection.
%
\vspace{3mm}
\begin{figure}[h]
\begin{center}
\begin{circuitikz}[line width=0.7pt, line join=round]
  \draw (0,1.5)
  to [open, *-*] (0,0);
  \draw (0,1.5)
  to [open, v^=$v_s$] (0,0);
  \draw (0,1.5)
  to [R] (1.5,1.5)
  to [R] (3,1.5)
  to [R, l_=$R_x$] (3,0)
  to [R] (1.5,0)
  to [R] (0,0);
  \draw (3,1.5) to [open, v^=$v_x$] (3,0);
  %
  \draw (5,0) to [open, *-*] (5,1.5);
  \draw (5,1.5)
  to [short, i_=$i_s$, *-*] (6,1.5)
  to [R, *-*] (6,0)
  to [short, *-*] (5,0);
  \draw (6,1.5) -- (7,1.5)
  to [R,l_=$R_x$, *-*] (7,0) -- (6,0);
  \draw (7,1.5) -- (8,1.5)
  to [R,*-*] (8,0) -- (7,0);
  \draw (8,1.5) to [open, v^=$v$] (8,0);
\end{circuitikz}
\caption{A simple voltage and current division circuits.}
\end{center}
\end{figure}
\vspace{-4mm}
%
Furthermore, the $\Delta$-Y transformation transforms an interconnection of 3 elements between the two types of junction, via
\begin{equation}
    R_i = \begin{cases}
        (R_j'R_k') \,/\, (R_i' + R_b' + R_c') & \Delta \to Y \\[0.5mm]
        (R_i'R_j' + R_j'R_k' + R_k'R_i') \,/\, R_i' & Y \to \Delta
\end{cases}
\end{equation}
respectively.

\section{Nodal Analysis}
The method of nodal analysis is most commonly used when no node in the circuit connects more than three branches. It is done procedurally as follows:
\begin{enumerate}
    \item Select the node with the most branches as the reference node, typically the bottom-most node.

    \item Define the node voltages $v_i$, which are the voltage rises across a branch from the reference node to another node $i$.

    \item For each non-reference node, generate a KCL equation
    \begin{equation}
        i_a : \sum i_a = \sum (v_a - v_s)/R_i = 0
    \end{equation}
    to solve for $v_a$.

    \item If a voltage source is between two nodes, a supernode can be formed in which the nodes are related by $v_i = v_j + v_s$.
\end{enumerate}

\section{Mesh Analysis}
Consequently, the mesh analysis method is used for nodes with $>3$ branches. It is done as follows:
\begin{enumerate}
    \item Assign mesh current $i$ directions around each loop, conventionally clockwise.

    \item For all nodes, develop individual KCL equations.

    \item For each mesh, generate a KVL equation \begin{equation}
        v_a : \sum v_a = \sum (i_a - i_i)R_i = 0
    \end{equation}
    where $i_i$ is any other mesh current passing through $R_i$ if applicable.

    \item Solve for the branch currents, letting $i_a = i_\alpha$. If a branch is shared between mesh currents, $i_\alpha = i_a - i_b$.
\end{enumerate}

\section{Source Transformations}
The source transformation technique allows for a voltage source $v_s = i_s Z$ in series with an impedance $Z$ to be transformed into an equivalent current source $i_s$ in parallel with $Z$, and vice versa. These transformations are known as a Thevenin and Norton equivalent circuit, respectively.

\bigskip
A result of this theorem is that an impedance that is in parallel with $v_s$ or in series with $i_s$ has no effect at the terminals and thus can be removed from the circuit.
%
\vspace{3mm}
\begin{figure}[h]
\begin{center}
\begin{circuitikz}[line width=0.7pt, line join=round]
  \draw (0,0)
  to [V, v_=$v_s$, invert] (0,1.5)
  to [R] (2.5,1.5)
  to [open, *-*, v=$ $] (2.5,0)
  -- (0,0);
  %
  \draw (5,0)
  to [I, l_=$i_s$] (5,1.5)
  -- (7.5,1.5)
  to [open, *-*, v=$ $] (7.5,0)
  -- (5,0);
  \draw (6.5, 0)
  to [R] (6.5,1.5);
\end{circuitikz}
\caption{Thevenin and Norton equivalent circuits}
\end{center}
\end{figure}
\vspace{-4mm}
%

\section{Thevenin's Theorem}
Thevenin's theorem is useful in finding the current or voltage of the terminals of a load $Z_L$. It defines the method to calculate $V_T$, by first replacing $Z_L$ with an open circuit and finding the voltage drop $V_T$ across the open circuit, typically via voltage division.

\bigskip
\textbf{Method 1 (Mixed Sources)}: The current $i_s$ is found by replacing $Z_L$ with a short circuit and calculating the resulting current. This leads to the Thevenin impedance being
\begin{equation}
    Z_T = V_T / i_s
\end{equation}

\bigskip
\textbf{Method 2 (Ind. Sources)}: Deactivate all independent sources, then calculate the impedance $Z_{eq} = Z_T$ of the resulting network.

\bigskip
\textbf{Method 3 (Dep. Sources)}: Deactivate all independent sources, then apply a test source between the terminals. This gives $V_s = V_T$.
%
\vspace{3mm}
\begin{figure}[h]
\begin{center}
\begin{circuitikz}[line width=0.7pt, line join=round]
  \draw (0,0)
  to [twoport, t=$N$] (0,1.5)
  -- (2.5,1.5)
  to [twoport, t=$Z_L$] (2.5,0)
  -- (0,0);
  \draw (2,1.5)
  to [open, *-*, v_=$V_T$] (2,0);
  %
  \draw (5,0)
  to [V, v=$V_T$, invert] (5,1.5)
  to [twoport, t=$Z_L$] (7,1.5)
  -- (7.5,1.5)
  to [twoport, t=$Z_L$] (7.5,0)
  -- (5,0);
  \draw (7,1.5)
  to [open, *-*, v_=$V_L$] (7,0);
\end{circuitikz}
\caption{A typical circuit converted to a Thevenin circuit}
\end{center}
\end{figure}
\vspace{-4mm}
%

\section{Superposition}
The principle of superposition states suppressing all but one source sequentially and summing the values is allowed. To suppress a voltage source is to replace it with a short circuit and a current source an open circuit.

\bigskip
Specifically, for a circuit with $n$ independent sources, a maximum of $n$ superposition circuits may be created with one source per circuit. The total current or voltage of the original circuit is thus $x = \sum x_i$.




\end{document}
