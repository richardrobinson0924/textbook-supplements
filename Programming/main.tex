\documentclass{tufte-book}
\usepackage[utf8]{inputenc}
\usepackage[english]{babel}
\usepackage{listings}

\setlength{\parindent}{0pt}
\title{C \& Unix}
\author{Richard Robinson}

\lstset{language=c}
\lstset{%
  aboveskip=3mm,
  belowskip=3mm,
  showstringspaces=false,
  basicstyle={\small\ttfamily},
  numbers=none,
  numberstyle=\tiny\color{gray},
  captionpos=none,
  title=\lstname,
  keywordstyle=\bfseries\color{magenta},
  commentstyle=\color{green!50!black},
  stringstyle=\color{red},
  breaklines=true,
  breakatwhitespace=true,
  tabsize=4
}

\begin{document}
\frontmatter
\maketitle
\setlength{\parindent}{0pt}
\mainmatter

\chapter{Intro to C}

\section{The Basics}

C is a command line language. The basic syntax of any C program code is
\begin{lstlisting}
    #include <stdio.h>
    int main(void) {
        /* Program */
    }
\end{lstlisting}
If the program takes command line arguments, then the function instead becomes
\begin{lstlisting}
    int main(int argc, char *argv[])
\end{lstlisting}
A string is treated as an array of characters where the last element is \lstinline{\0} and is declared by \lstinline{char *}. For a constant variable, use \lstinline{const} as a prefix in the declaration. Alternatively,
\begin{lstlisting}
    #define NAME value
\end{lstlisting}
may be used. Numbers in base 8 and 16 are prefixed by \lstinline{0} and \lstinline{0x}, respectively. An enumeration is an array of constants, with syntax
\begin{lstlisting}
    enum type { c1 = value, c2, c3 = value ...};
    int x = c2
\end{lstlisting}
where the values are optional.

\section{Formatting Output}
The \lstinline{printf()} function outputs text.
\marginnote{For example, \lstinline{printf("\%6.1f string", 42.42)} outputs \lstinline{    42.4 string}.}
In the output string, \lstinline{\%x.yf} formats the respective float such that there are $x$ and $y$ characters to the left and right of the decimal point, respectively.

\bigskip
To cast a variable is to change it from one type to another, via
\begin{lstlisting}
    int var = (int)(var2);
\end{lstlisting}
where \lstinline{var2} need not be an int. Additionally, use \lstinline{x.f} to convert the int $x$ to type float.

\section{Conditionals}
The ternary condition \lstinline{x = exp1 ? a : b} is equivalent to
\marginnote{The code \lstinline{c = (a > b) ? a : b} is equivalent to \lstinline{c = max(a,b)}.}
\begin{lstlisting}
    if (exp1) x = a;
    else x = b;
\end{lstlisting}
The switch statement is a condensed if-else-if-else... statement, with the general form
\begin{lstlisting}
    switch (exp) {
        case 1: statements; break;
        case 2: statements; break
        ...
        default: statements
    }
\end{lstlisting}
Note \lstinline{break} is optional. If not included, the next case may be executed. Otherwise, it branches out of a statement or loop (except for if-else).

\bigskip
The \lstinline{continue} command is used in a nested if statement to skip a certain iteration(s) of the outer loop. It is often used to skip certain elements, as follows:
\begin{lstlisting}
    while (exp) {
        if (condition) continue; /* skip statements */
        statements;              /* if !condition... */
    }
\end{lstlisting}


\end{document}
