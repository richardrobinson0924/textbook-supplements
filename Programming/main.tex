\documentclass{tufte-book}
\usepackage[utf8]{inputenc}
\usepackage[english]{babel}
\usepackage{amsmath}
\usepackage{amssymb}
\usepackage{listings}
\usepackage{eso-pic}
\usepackage{empheq}
\usepackage[dvipsnames]{xcolor}
\usepackage{graphicx}
\graphicspath{{../images/}}

% cover photo
\newcommand\BackgroundPic{%
\put(0,0){%
\parbox[b][\paperheight]{\paperwidth}{%
\vfill
\centering
\includegraphics[height=\paperheight,%
keepaspectratio]{keyboard}%
\vfill
}}}

\setlength{\parindent}{0pt}
\colorlet{darkgray}{white!85!black}
\title{C, Unix, \& Java}
\author{Richard Robinson}

\lstset{language=c}
\lstset{%
  aboveskip=3mm,
  belowskip=3mm,
  showstringspaces=false,
  basicstyle={\small\ttfamily},
  numbers=none,
  numberstyle=\tiny\color{gray},
  captionpos=none,
  title=\lstname,
  keywordstyle=\bfseries\color{magenta},
  commentstyle=\color{green!50!black},
  stringstyle=\color{red},
  breaklines=true,
  breakatwhitespace=true,
  tabsize=4
}

\begin{document}
\AddToShipoutPicture*{\BackgroundPic}
\frontmatter
\maketitle
\setlength{\parindent}{0pt}
\mainmatter

\chapter{Intro to C}

\section{The Basics}

C is a command line language. The basic syntax of any C program code is
\begin{lstlisting}
    #include <stdio.h>
    int main(void) {
        /* Program */
    }
\end{lstlisting}
If the program takes command line arguments, then the function instead becomes
\begin{lstlisting}
    int main(int argc, char *argv[])
\end{lstlisting}
A string is treated as an array of characters where the last element is \lstinline{\0} and is declared by \lstinline{char *}. For a constant variable, use \lstinline{const} as a prefix in the declaration. Alternatively,
\begin{lstlisting}
    #define NAME value
\end{lstlisting}
may be used. Numbers in base 8 and 16 are prefixed by \lstinline{0} and \lstinline{0x}, respectively. An enumeration is an array of constants, with syntax
\begin{lstlisting}
    enum type { c1 = value, c2, c3 = value ...};
    int x = c2
\end{lstlisting}
where the values are optional.

\section{Formatting Output}
The \lstinline{printf()} function outputs text.
\marginnote{For example, \lstinline{printf("\%6.1f string", 42.42)} outputs \lstinline{    42.4 string}.}
In the output string, \lstinline{\%x.yf} formats the respective float such that there are $x$ and $y$ characters to the left and right of the decimal point, respectively.

\bigskip
To cast a variable is to change it from one type to another, via
\begin{lstlisting}
    int var = (int)(var2);
\end{lstlisting}
where \lstinline{var2} need not be an int. Additionally, use \lstinline{x.f} to convert the int $x$ to type float.

\section{Conditionals \& Loops}
The ternary condition \lstinline{x = exp1 ? a : b} is equivalent to
\marginnote{The code \lstinline{c = (a > b) ? a : b} is equivalent to \lstinline{c = max(a,b)}.}
\begin{lstlisting}
    if (exp1) x = a;
    else x = b;
\end{lstlisting}
The switch statement is a condensed if-else-if-else... statement, with the general form
\begin{lstlisting}
    switch (exp) {
        case 1: statements; break;
        case 2: statements; break
        ...
        default: statements
    }
\end{lstlisting}
Note \lstinline{break} is optional. If not included, the next case may be executed. Otherwise, it branches out of a statement or loop (except for if-else).

\bigskip
The \lstinline{continue} command is used in a nested if statement to skip a certain iteration(s) of the outer loop. It is often used to skip certain elements, as follows:
\begin{lstlisting}
    while (exp) {
        if (condition) continue; /* skip statements */
        statements;              /* if !condition... */
    }
\end{lstlisting}

\section{Functions}

A function must be declared before it is called. The general format is:
\begin{lstlisting}
    int name(int, int);
    /* code that calls function */
    int name(int x, int y) {
        /* code goes here */
        return z;
    }
\end{lstlisting}
The argument type must match the function type. A function who's type is \lstinline{void} has no return value.
\marginnote{Note an argument of type \lstinline{char[]} may be used in an \lstinline{int} function.}

\bigskip
The \lstinline{assert(condition)} function terminates a program if the condition is false. It is exclusively used for debugging.

\chapter{Software Design in C}

\section{Scope}

Variables prefixed with \lstinline{static} have their scope restricted to the file they are in. Header files imported via
\begin{lstlisting}
    #include "filename.h"
\end{lstlisting}
are used to store declarations and code shared over multiple source files, and effectively act as a macro. A pointer is declared using \lstinline{*} where \lstinline{&} is the address operator. The syntax is used as
\begin{lstlisting}
    int i;
    int *j = &i;
\end{lstlisting}
Pointers are mainly used within functions; any changed made to the local variables within the function will not affect nor change the variables of the calling function, used as:
\begin{lstlisting}
    func(&a, &b);
    void func(int* x, int* y) {
        /* code */
    }
\end{lstlisting}
Additionally, \lstinline{*var} is effectively \lstinline{var[]}. In this case, \lstinline{var} is \lstinline{&var[0]} when used in an expression, and \lstinline{s[i]} is \lstinline{*(s+i)}.

\bigskip
An array can contain pointers with syntax \lstinline{int *arr[N]}.

\section{Arrays}
\marginnote{The line \lstinline{char str[] = "string"} is equivalent to \lstinline{char str[] = \{'s','t',...\}}.}
An array whom's size is larger than the number of elements has its remaining elements as 0. Size need not be specified. The number of elements for such array is
\begin{lstlisting}
    num_elements = sizeof(arr) / sizeof(arr[0]);
\end{lstlisting}
A function may return a pointer, which would have the form
\begin{lstlisting}
    int* func(void) {
        return &var;
    }
\end{lstlisting}
Like any array, a string may be indexed at any element. An important standard function is
\begin{lstlisting}
    char *strstr(const char *s, const char *t)
\end{lstlisting}
which searches for \verb|t| in \verb|s|. When indexing an array of strings, use \lstinline|char *str[] = {"string"}|. A for loop must be used to then iterate through each char in each string. Alternatively,
\begin{lstlisting}
    do {
        c = getchar();
        /* code */
    } while (c != '\0' && c != '\n');
\end{lstlisting}
may be used.

%---------------------------------------
% UNIX
%---------------------------------------
\lstset{language=bash}

\chapter{Intro to Unix}

\section{Basic Syntax}
The vertical bar character is used for sending the output of a command as the input to another command, with syntax
\begin{lstlisting}
    fromcommand | tocommand
\end{lstlisting}
In Unix, the main directories are as follows:
\begin{itemize}
    \item \verb|/bin|: Contains executable binaries users may run.
    \item \verb|/dev|: \verb|/etc|: Special and administrative files.
    \item \verb|/lib|: Central library storage for files commonly used elsewhere.
    \item \verb|/lostfound|, \verb|/tmp|: Extra files.
\end{itemize}
The syntax of a Unix command is
\begin{lstlisting}
    $ Name [-op1] [-op2] [Arg]
\end{lstlisting}
where \verb|op| are options.


\end{document}
