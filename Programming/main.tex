\documentclass{tufte-book}
\usepackage[utf8]{inputenc}
\usepackage[english]{babel}
\usepackage{amsmath}
\usepackage{amssymb}
\usepackage{listings}
\usepackage{eso-pic}
\usepackage{empheq}
\usepackage[dvipsnames]{xcolor}
\usepackage{graphicx}
\graphicspath{{../images/}}


\setlength{\parindent}{0pt}
\title{C, Unix, \& Java}
\author{Richard Robinson}

\lstset{language=c}
\lstset{%
  aboveskip=3mm,
  belowskip=3mm,
  basicstyle={\small\ttfamily},
  captionpos=none,
  keywordstyle=\bfseries\color{magenta},
  commentstyle=\color{green!50!black},
  stringstyle=\color{red},
  breaklines=true,
  breakatwhitespace=true,
  tabsize=4
}

\begin{document}
\maketitle
\setlength{\parindent}{0pt}
\begin{fullwidth}

\chapter{Intro to C}

\section{The Basics}

In C, there only exist multiline \lstinline{/* */} comments. As well, there is no garbage collection, classes, exception, nor strings. The basic C program syntax is
\begin{lstlisting}
    #include <stdio.h>
    main() {
        /* body */
        return 0;
    }
\end{lstlisting}
The \lstinline{#include} tag acts as a macro for whatever file is called. In C, the null character \lstinline{\0} is critical, as it is appended to all arrays of characters and therefore represents the end of a string.

\section{Basic I/O}
The basic input and output commands in C are \lstinline{getchar(), putchar(), printf()}, and \lstinline{scanf()};
\begin{itemize}
    \item The \lstinline{int getchar()} command reads one character at a time and returns \lstinline{EOF} at end of line, defined as -1.

    \item The \lstinline{int putchar(c)} commands returns \lstinline{c} to the standard output or \lstinline{EOF} if an error occurs.

    \item The \lstinline{printf("str")} command outputs the string with optional arguments.

    \item The \lstinline{scanf("\%x", &x)} command reads the input, assigning \lstinline{\%x} to \lstinline{x}. The \lstinline{x} type is a hexadecimal integer.
\end{itemize}
To parse a string by character in C, a while loop of the form
\begin{lstlisting}
    int c;
    while ((c = getchar()) != EOF) { }
\end{lstlisting}
is used. Other types in addition to \lstinline{x} are \lstinline{c, d, f, lf, s}. The basic data types in C are \lstinline{char} (8 bits), \lstinline{int} (16/32 bits), \lstinline{float} (4 bytes), and \lstinline{double} (8 bytes). A string is an array of characters such ended by null, with syntax
\begin{lstlisting}
    char varname[] = "string"; /* or */
    char *varname = "string" /* or */
    char varname[] = {'chr1', 'chr2', ..., '\0'}
\end{lstlisting}
Because of the nature of strings, \lstinline{printf()} is equivalent to
\begin{lstlisting}
    int printf(char *format, args);
\end{lstlisting}

\section{Printf and Scanf}
The format code \lstinline{"\%0n.df"} represents the following characteristics:
\begin{itemize}
    \item The \lstinline{n} is the total allotment of characters for the number, printing the characters from the right.
    \item The \lstinline{0} replaces all the leading whitespace characters allotted before the number with 0.
    \item The \lstinline{.d} is the number of decimal places to be included; the decimal point counts as a character for \lstinline{n}.
    \item The \lstinline{f} represents the number is of type float.
\end{itemize}
The default number of decimal places is 6. As well, a negative \lstinline{n} value adds whitespace to the right instead of left.

\bigskip
To read an integer from the input, the syntax
\begin{lstlisting}
    int num; scanf("%d", &num);
\end{lstlisting}
is used. The function stops reading upon EOF or a failed input occurs. Additionally, it returns the number of successfully matched inputs, and returns 0 for an error.

\bigskip
Addition, file-wise scanning and printing are given via
\begin{itemize}
    \item \lstinline{prog < infile}: prog reads chars from infile
    \item \lstinline{prog > infile}: prog writes chars to infile
    \item \lstinline{prog1 | prog2}: input of prog2 is output of prog1
\end{itemize}

\end{fullwidth}
\end{document}
